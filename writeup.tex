\documentclass{article}
\usepackage[utf8]{inputenc}
\usepackage{enumerate}

\begin{document}

\title{Programming Project Writeup}
\author{Kay Toma, Laura Harker, Anna Pham, Jennifer Yu}

\maketitle

\begin{enumerate}
\item FindUsOnTinder
\item cs170-jp
\item Kay Toma, Laura Harker, Anna Pham, Jennifer Yu
\item Our general approach was to modify the Lin-Kernighan algorithm as it was stated to be one of the best algorithms to solve the Euclidean Traveling Salesman problem. We used the 2-opt and 3-opt versions. We generate a psuedo-random path and improve our solution from there. We also looked for valid paths in terms of coloring pattern RB, RRBB, and RRRBBB. 
\item Our general approach was following a paper we found online. We first created a Hamiltonian path in the form of a diamond circuit. The diamond circuit could have two paths--north to south and east to west. We could either choose east-west or north-south as the ``true'' path and we hoped to trick algorithms into traveling the north-south path because it was sometimes inexpensive but overall very costly. If anybody tried to run a greedy algorithm to solve the graph, they would end up with an inaccurately high answer. 
\item We use paths.py as a library. The bulk of the work can be run on 23opt.py. Running ``python 23opt.py'' on the command line will solve all 495 instances. 
\item We used the 1978 paper titled ``Some Examples of Difficult Traveling Salesman Problems'' by Papadimitriou and Steiglitz. 
\end{enumerate}